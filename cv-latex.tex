\documentclass[a4paper,10pt]{article}
\usepackage[utf8]{inputenc}
\usepackage[T1]{fontenc}
\usepackage[french]{babel}
\usepackage{geometry}
\usepackage{xcolor}
\usepackage{enumitem}
\usepackage{titlesec}
\usepackage[hidelinks]{hyperref}
\usepackage{fontawesome5}
\usepackage{multicol}

\geometry{margin=1.5cm}
\definecolor{myblue}{HTML}{0B3954}
\definecolor{mygray}{HTML}{4B4B4B}
\definecolor{myorange}{HTML}{F2542D}

% Section formatting
\titleformat{\section}{\color{myblue}\Large\bfseries}{}{0em}{}
\titleformat{\subsection}[runin]{\bfseries\color{myorange}}{}{0em}{}[ -- ]

\pagestyle{empty}

\begin{document}

%==== HEADER ====%
\begin{center}
    {\LARGE \textbf{Younes EL MIR}} \\[4pt]
    \href{mailto:younes.elmir@email.com}{younes.elmir@email.com} $\cdot$ 06 12 34 56 78 $\cdot$ Paris, France \\
    \href{https://linkedin.com/in/youneselmir}{linkedin.com/in/youneselmir} $\cdot$ \href{https://github.com/younes-devops}{github.com/younes-devops}
\end{center}

%==== TITLE ====%
\begin{center}
    \textcolor{mygray}{\large Ingénieur DevOps \textbar{} Cloud \textbar{} CI/CD \textbar{} SRE}
\end{center}

space{0.5em}

%==== COMPÉTENCES ====%
\section*{Compétences Techniques}
\begin{multicols}{2}
\begin{itemize}[leftmargin=*]
    \item \textbf{CI/CD} : GitLab CI, GitHub Actions, Jenkins, ArgoCD
    \item \textbf{Cloud} : AWS (EC2, S3, EKS), GCP, Azure
    \item \textbf{IaC} : Terraform, Ansible, Pulumi
    \item \textbf{Containers} : Docker, Kubernetes, Helm, Kustomize
    \item \textbf{Monitoring} : Prometheus, Grafana, ELK, Loki
    \item \textbf{Scripting} : Bash, Python, Groovy
    \item \textbf{Sécurité} : Vault, SOPS, DevSecOps
    \item \textbf{Outils} : Linux, Git, PostgreSQL, Redis
\end{itemize}
\end{multicols}

%==== EXPÉRIENCES ====%
\section*{Expériences Professionnelles}

\subsection{Ingénieur DevOps, Alten \hfill Oct. 2022 -- Présent}
\textit{Projet bancaire – Paris}
\begin{itemize}[leftmargin=*]
    \item Mise en place de pipelines GitLab CI pour 20 microservices.
    \item Déploiement de clusters EKS avec Terraform.
    \item Monitoring avec Prometheus et Grafana.
    \item Gestion des secrets avec Vault, stratégie GitOps avec ArgoCD.
\end{itemize}

\subsection{DevOps Engineer, Capgemini Engineering \hfill Mars 2021 -- Août 2022}
\textit{Projet IoT – Lyon}
\begin{itemize}[leftmargin=*]
    \item Provisioning automatique via Terraform + Ansible.
    \item Migration Docker Swarm vers Kubernetes.
    \item Collecte de logs avec ELK + Filebeat.
\end{itemize}

\subsection{Freelance DevOps \hfill Depuis 2021}
\textit{Remote}
\begin{itemize}[leftmargin=*]
    \item Pipelines GitHub Actions pour sites e-commerce.
    \item Clusters Kubernetes locaux avec k3s.
    \item Audit sécurité : IAM, hardening, ports.
\end{itemize}

%==== PROJETS ====%
\section*{Projets Techniques}
\subsection{CI/CD Kubernetes avec GitLab}
\begin{itemize}[leftmargin=*]
    \item Pipelines multistages avec Helm, review apps éphémères.
    \item Secrets chiffrés avec SOPS + GCP KMS.
\end{itemize}

\subsection{Infra GitOps Cloud (Open Source)}
\begin{itemize}[leftmargin=*]
    \item ArgoCD + Helmfile, cluster EKS, services auto-déployés.
    \item Scripts Python d'audit IAM et alertes sécurité.
\end{itemize}

%==== FORMATION ====%
\section*{Formation}
\textbf{Diplôme d'ingénieur} – Informatique et Systèmes Embarqués \
ENSIM / ISEN – 2017--2022  \\
\textbf{Certifications} : AWS SA Associate, CKA, Terraform Associate

%==== LANGUES ====%
\section*{Langues}
Français (natif), Anglais (professionnel – C1)

%==== VEILLE ====%
\section*{Veille et Contributions}
\begin{itemize}[leftmargin=*]
    \item Articles Medium : \textit{GitOps vs DevOps}, \textit{Migration K8s sans downtime}
    \item Conférencier : DevOps D-Day 2024, Paris Meetup
    \item Contributions GitHub sur ArgoCD, Helm, Terraform
\end{itemize}

\end{document}


********************************************************
Voici un exemple complet de CV pour un ingénieur DevOps postulant à un CDI, structuré pour être percutant, moderne et orienté résultats :
Younes EL MIR

📍 Paris, France | 📞 06 12 34 56 78 | 📧 younes.elmir@email.com | 💼 linkedin.com/in/youneselmir | 🖥️ GitHub : github.com/younes-devops
🧑‍💻 Ingénieur DevOps

Passionné par l’automatisation, la fiabilité des infrastructures et le CI/CD, j’accompagne les équipes de développement dans la mise en production rapide, sécurisée et scalable de leurs applications. Force de proposition sur les environnements Cloud (AWS, GCP) et les pratiques GitOps, je maîtrise l’ensemble de la chaîne de livraison applicative.
✅ COMPÉTENCES CLÉS

    CI/CD : GitLab CI, GitHub Actions, Jenkins, ArgoCD

    Cloud : AWS (EC2, S3, IAM, EKS), GCP, Azure (notions)

    Infrastructure as Code : Terraform, Ansible, Pulumi

    Containers & Orchestration : Docker, Kubernetes, Helm, Kustomize

    Monitoring & Observabilité : Prometheus, Grafana, ELK, Loki, Datadog

    Scripting & Automatisation : Bash, Python, Groovy

    Sécurité & Compliance : Vault, SOPS, IAM, MFA, DevSecOps

    Outils : Linux, Git, Nginx, PostgreSQL, Redis, Kafka, RabbitMQ

    Méthodologies : Agile/Scrum, GitOps, SRE, DevSecOps, DRP/HA

💼 EXPÉRIENCES PROFESSIONNELLES
🔧 Ingénieur DevOps – Alten

Paris | CDI | Octobre 2022 – Présent
Contexte : Projet de refonte complète des pipelines CI/CD pour un client dans le secteur bancaire.

    Mise en place de pipelines GitLab CI pour 20 microservices (tests, build, déploiement sur Kubernetes).

    Déploiement de clusters EKS via Terraform, avec stratégie de blue/green deployment.

    Création de dashboards Grafana/Prometheus pour le monitoring des pods, CPU/mem, et alerting.

    Intégration de Vault pour la gestion des secrets et rotation automatique des clés AWS.

    Réduction des temps de déploiement de 45% via une approche GitOps (ArgoCD + Helm).

    Participation aux astreintes (SRE) : réduction du MTTR de 35% en automatisant la détection des anomalies.

Tech Stack : GitLab CI, Kubernetes (EKS), Helm, ArgoCD, Terraform, Vault, AWS, Prometheus, Grafana
🧪 DevOps Engineer – Capgemini Engineering (stage de fin d'études puis CDD)

Lyon | Mars 2021 – Août 2022
Projet : Déploiement d’une plateforme IoT sécurisée dans un environnement cloud hybride.

    Déploiement automatique d’environnements de test via Terraform + Ansible.

    Création de playbooks Ansible pour configurer 50+ VMs avec installation de Docker et monitoring.

    Déploiement des apps IoT via Docker Swarm, puis migration vers Kubernetes pour scalabilité.

    Rédaction de scripts Bash pour automatiser les backups et restaurations (minio + PostgreSQL).

    Mise en place d’un cluster ELK pour la collecte de logs des devices IoT (logstash + filebeat).

🖥️ Freelance DevOps (missions ponctuelles)

Remote | Depuis 2021

    Création d’un pipeline CI/CD GitHub Actions pour un site e-commerce (tests Cypress + déploiement AWS).

    Déploiement Kubernetes local via k3s, configuration Ingress et cert-manager pour HTTPS.

    Audit de sécurité d’infrastructure (IAM policies, ouvertures de ports, hardening Linux).

🎓 FORMATION

🎓 Diplôme d’Ingénieur – Informatique & Systèmes Embarqués
ENSIM / ISEN – 2017 – 2022
Spécialisation DevOps et systèmes distribués

📚 Certifications :

    ✅ AWS Certified Solutions Architect – Associate

    ✅ CKA – Certified Kubernetes Administrator

    ✅ Terraform Associate (HashiCorp)

📁 PROJETS TECHNIQUES MARQUANTS
🚀 CI/CD Multi-environnement avec GitLab, Docker et Kubernetes (projet personnel)

    Création d’un monorepo avec GitLab CI, utilisant des runners Docker et Helm pour le déploiement.

    Pipeline : lint → test → build → push image → review app (namespace temporaire) → prod.

    Gestion des secrets via GitLab CI + SOPS + GCP KMS.

🔒 Infrastructure GitOps full Cloud (projet open source)

    Mise en place d’un repo GitOps avec ArgoCD et Helmfile.

    Cluster K8s EKS (Terraform), déploiement automatisé de services (Traefik, Keycloak, Loki).

    Scripts Python pour audit automatique des IAM policies et check sécurité hebdo.

💬 LANGUES

    🇫🇷 Français : Langue maternelle

    🇬🇧 Anglais : Professionnel (C1)

📡 VEILLE TECHNOLOGIQUE & CONTRIBUTIONS

    Participation active à la communauté DevOps sur GitHub (pull requests, issues sur Helm et ArgoCD).

    Rédaction d’articles techniques sur Medium : « GitOps vs DevOps : différences et complémentarité », « Comment migrer un cluster K8s sans downtime ».

    Conférencier occasionnel (DevOps D-Day 2024, Paris DevOps Meetup).

Souhaite-tu que je t’en fasse une version PDF prête à l’emploi, ou bien tu préfères un format Word / LaTeX ? Je peux aussi t’aider à rédiger une lettre de motivation sur mesure pour ton CDI cible.
You said:
un latex bien beau et attrient
ChatGPT said:

Voici un CV LaTeX élégant et moderne pour un poste d’ingénieur DevOps. Tu peux l’exporter en PDF en le compilant avec Overleaf (ou tout compilateur LaTeX local). Si tu veux, je peux aussi t’ajouter une photo, une lettre de motivation personnalisée ou l’adapter à une offre précise. Tu veux que je m’en charge ?


ChatGPT can make mistakes. Check important info.
